Blockchain-based Federated Learning has emerged as an alternative to Federated Learning in order to solve some of its issues. On one hand, it eliminates the central orchestrator, allowing multiple distributed clients to cooperate on the same Machine Learning model, without a single point of failure in the network. On the other hand, it facilitates aspects such as traceability, auditability, authentication and persistency, that, together, improve the safety of the whole process and allow for new verification mechanisms in order to detect bad agents.

In these systems, it is common to score each client's submission in order to determine if it is a good contribution to the global model. With Blockchain-based Federated Learning being increasingly adopted in IoT networks, where low powered devices with low resources are the norm, it is important to ensure that certain aspects of the system consume the least amount of resources. The current literature has very little information regarding how different aspects impact the system. Additionally, there is no publicly available framework that can be used to implement a Blockchain-based Federated Learning system.

In this thesis, we design and implement the first modular open-source framework for Blockchain-based Federated Learning using Ethereum and TensorFlow. This framework can be easily adapted to support multiple architectures, as well as different scoring, aggregation and privacy techniques. With this framework, we proceed to do the first known analysis of how different aspects of Blockchain-based Federated Learning, such as consensus algorithms, participation selection techniques and scoring techniques, impact the accuracy, execution time and communication and computation costs. Additionally, the same analysis is done per each scoring technique regarding the impact of the number of clients and privacy techniques. Finally, we also provide a proof of concept of how the framework can be adapted to support, not only Horizontal Federated Learning, but also Vertical Federated Learning.