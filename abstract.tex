Federated Learning is a technique that allows multiple clients, in different locations, to cooperate on the same Machine Learning model without sharing their own data with each other in order to preserve privacy of the data. Currently, most FL networks include a central server that coordinates the entire process and aggregates the model parameters from each client into a single model. This central coordinator is also a single point of failure in the network, since it always needs to be online in order to train the model. Recently, Blockchain-based Federated Learning techniques have been proposed to replace the central server and replace it by a distributed ledger, eliminating the single point of failure.

Even though Blockchain-based Federated Learning techniques were created to solve the single point of failure issue, they can also bring other problems to the table, such as communication costs, computation costs and therefore energy consumptuion. 




% One important factor is the communication and computation costs and therefore the energy consumption. On one hand, FL has been applied more and more to IoT networks, where low powered devices with low resources are the norm. In this scenario, it is important to ensure that the entire training process and global model updates consumes the least amount of energy. On the other hand, BFL mining is usually performed by other devices that should also take into consideration sustainability. In addition, FL is being applied to systems of real-time analysis, where low latency is a requirement. Thus, this research aims to answer what is the impact of different BFL properties on communication and computation costs, as well as accuracy and convergence time, when compared to locally trained models. Additionally, the following aspects will be taken into consideration: data partition types, consensus algorithms, model creation delays, model update frequencies and storage of model parameters.


% Even though BFL systems promise to solve some issues, other issues come up with this system. One important factor is the communication and computation costs and therefore the enery consumption. On one hand, FL has been applied more and more to IoT networks, where low powered devices with low resources are the norm. In this scenario, it is important to ensure that the entire training process and global model updates consumes the least amount of energy. On the other hand, BFL mining is usually performed by other devices that should also take into consideration sustainability. In addition, FL is being applied to systems of real-time analysis, where low latency is a requirement. Thus, this research aims to answer what is the impact of different BFL properties on communication and computation costs, as well as accuracy and convergence time, when compared to centralized/regular FL. Additionally, the following aspects will be taken into consideration: data partition types, consensus algorithms, model creation delays, model update frequencies and storage of model parameters.