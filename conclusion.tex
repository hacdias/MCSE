\todo{}
    
\section{General Conclusions}\label{conclusions:general}

% Consensus Algorithms
%     Computation Costs: PoW >> QBFT = PoA
%     Communication Costs: QBFT >> PoW = PoA
%     PoA can be considered less decentralized, which can be an issue in public networks.
    
% Participant Selection Methods
%     Computation and Communication Costs: PoW = QBFT = PoA
%     Fairness: Random > FCFS

% Scoring Technique
%     Computation and Communication Costs:
%         Server: Multi-KRUM > BlockFlow and Marginal Gain
%         Clients: BlockFlow and Marginal Gain > Multi-KRUM
%     Accuracy: Marginal Gain > Multi-KRUM > BlockFlow

% Decision Tree For Scoring Technique:
%     Low-Powered
%         Yes: 1. Multi-KRUM
%         No: 1. Marginal Gain, 2. BlockFlow

\section{Contributions}\label{conclusions:contributions}

\begin{enumerate}
    \item Designed and implemented the first open-source modular framework for Blockchain-based Federated Learning that can be easily adapted to support new scoring, aggregation and privacy techniques.
    
    \item Provided the first comparative study of how different aspects of Blockchain-based Federated Learning, namely consensus algorithms, participant selection techniques and scoring techniques, impact the accuracy, execution time and communication and computation costs.
    
    \item Provided the first comparative study of how the number of training devices and different degrees of privacy impact the accuracy, execution time and communication and computation costs of different scoring techniques.
    
    \item Designed and implemented the first open-source proof-of-concept of a Blockchain-based Vertical Federated Learning.
\end{enumerate}

\section{Future Work}\label{conclusions:future_work}