In this chapter, the different concepts and definitions regarding Machine Learning, Federated Learning and Blockchain are introduced. These can be regarded as necessary prior knowledge for understanding the remaining chapters of this report.

\section{Machine Learning}

Machine Learning (ML) is a sub-field of Artificial Intelligence that uses algorithms and statistical models in order to detect relevant patterns based on prior experience \cite{geron_2019}. By giving models data, they can learn and adapt without explicit instructions. The data is usually structured as vectors in a multi-dimensional space, such that each vector is an \textit{instance} and each dimension is a \textit{feature}.

The remaining of this sub-section will shortly introduce the main categories of Machine Learning and the problem formulation of classification problems.

\subsection{Categories of Machine Learning}

There are three main categories of Machine Learning algorithms: supervised learning, unsupervised learning and reinforcement learning \cite{geron_2019}. Each category performs different tasks in different types of data.

\paragraph{Supervised Learning} In supervised learning, algorithms build mathematical models using a set of input samples, which are vectors from a feature space, and expected outputs for each sample. This type of data is known as labeled data, as there is a label for each sample. During a process called training, algorithms feed the model the input samples and improve the model by comparing the model output with the expected outputs. Supervised learning problems can be divided into \textit{regression problems}, if the output is a continuous variable, or \textit{classification problems}, if the output is a discreet variable.

\paragraph{Unsupervised Learning} In unsupervised learning, in contrast to supervised learning, algorithms build mathematical models using unlabeled data. Therefore, there are no expected outputs that can be directly compared to the model's output. Some common problems solved by unsupervised learning are \textit{clustering}, \textit{dimensionality reduction} and \textit{anomaly detection}.

\paragraph{Reinforcement Learning} In reinforcement learning, models must learn the best action to take according to the current environment, receiving either \textit{rewards} or \textit{penalties} if they perform a good, or bad action, respectively.

\subsection{Classification Problems}

TODO

\section{Federated Learning}

Federated Learning (FL) is a Machine Learning technique where a set of decentralised devices, each one with their own resources and data sets, collaborate on training a model, under the orchestration of a central server. During the training process, the local data sets are never shared between parties, only some representation of the weights of the model. This is a concept introduced by Google researchers in 2016 \cite{10.48550/arxiv.1602.05629}.

\subsection{Categories of Federated Learning}

\section{Blockchain}

\cite{nakamoto2009bitcoin}

% - **Consensus Mechanism**: ****the mechanisms used by the network to reach a consensus on the current state.
% - **Permission Types**
%     - **Public**: permissionless, no central authority
%     - **Private**: permissioned, controlled by one authority
%     - **Consortium:** permissioned, controlled by a group
% - **Platforms**
%     - Ethereum
%     - Hyperledger Fabric
%     - EOS
%     - Others
    
    
\section{Blockchain-enabled Federated Learning}