In Blockchain-based Federated Learning (BFL) systems, there are different algorithms that need to be taken into consideration when building the system. In this chapter, we go over these algorithms and their variations used in the literature.

\section{Consensus Algorithms}\label{related_work:consensus_algorithms}

One of the most important components of blockchain technology is the consensus algorithm. Consensus is the process of reaching an agreement on a single value among different distributed processes \cite{9347812}. These algorithms are designed to be reliable even on networks that have unreliable nodes. In blockchain, the consensus algorithm is used to reach consensus on the next block of the chain \cite{9079513}.

As it can be seen from \autoref{tab:platf_consensus}, various consensus algorithms have been used for BFL systems. Below is a summary of each of these consensus algorithms.

\begin{itemize}
    \item \textit{Proof of Work (PoW)} has been used for many years and some of its advantages and disadvantages are now clear. On one hand, it is a simple algorithm, for which proofs are hard to create, but easy to verify. Not only is it robust and proven to work, but the cost of attacking a PoW blockchain is extremely high. For an attack to be successful, it needs to control more than half of the network \cite{li_blockchain_2021}. On the other hand, PoW consumes extreme amounts of energy and it is hard to scale \cite{edwood_2020, li_blockchain_2021, ccaf}. In addition, authors of \cite{10.48550/arxiv.2112.07938} mentions the importance of analyzing the constraints and trade-offs of using PoW with BFL.

    \item \textit{Proof of Stake (PoS)}, in contrary to the PoW, does not require high computational resources from the blockchain nodes and therefore the energy consumption can be kept lower. In addition, it allows for fast throughput and blockchain nodes are incentivized to behave correctly through a rewarding system \cite{li_blockchain_2021}. On the other hand, some nodes may have excessive influence over the transaction verification process \cite{li_blockchain_2021}.
    
    \item \textit{Proof of Authority (PoA)} is a highly scalable consensus algorithm with high throughput \cite{binance_academy_2020}. However, one of the main criticisms on the PoA is that it is usually used with a small number of validators, that are manually chosen. Therefore, it has a lesser degree of decentralization. Consequently, PoA is not as common in public blockchain networks as it is for private networks \cite{bPoA}.

    \item \textit{Proof of Federated Learning (PoFL)} \cite{9347812, 10.48550/arxiv.2007.15145} and \textit{Proof of Quality (PoQ)} \cite{8843900} are both consensus algorithms that integrate the training process in order to reduce the resources and energy consumption. These are custom algorithms and as such are not readily available on public blockchain platforms. To use them, developers either need to implement their custom blockchain platforms or modify an existing blockchain platform.

    \item \textit{Practical Byzantine Fault Tolerance (PBFT)} allows for high consensus efficiency in high throughput networks \cite{li_blockchain_2021}. However, it will stop working properly if only 33\% or less nodes are running and it can also have high communication costs due to its three-phase protocol nature.

    \item \textit{Committee-based Consensus} is a group of consensus algorithms where a selected number of members from a committee are selected in order to achieve consensus in a fast way \cite{qu_blockchain-enabled_2022}. This is usually used on custom blockchain implementations with specific goals, such as minimizing communication costs, in mind \cite{9293091}.
\end{itemize}

\section{Model Parameter Storage}\label{related_work:param_storage}

Another important component of BFL systems is the location, where the model parameters are stored in order to be shared with the servers. According to the literature, the model parameters may either be stored on-chain, i.e., in the blockchain itself, or off-chain, i.e., in a separate storage provider \cite{10.48550/arxiv.2104.10501}.

With \textit{on-chain storage} \cite{9274451, baffle, demo, 8733825, 9524833, 8894364, 9184854, 8893114}, the smart contract stores the model parameters itself, which means that the parameters themselves will be stored in the blockchain. However, most blockchain platforms have a limit on how large a block can be and, consequently, the amount of data that can be stored per contract is limited \cite{9274451}. In these cases, smart contracts are chunked, i.e., a single contract is split into many different contracts that hold smaller chunks of the parameters \cite{baffle}. In addition, this allows for the new model parameters to be directly calculated through the smart contract as the values are directly accessible  \cite{9274451}.
    
With \textit{off-chain storage} \cite{10.1145/3319535.3363256, 10.48550/arxiv.2011.07516, 8945913, 10.48550/arxiv.2202.02817, 10.48550/arxiv.2007.03856, 10.48550/arxiv.1910.12603, Peyvandi2022, 9170559}, the smart contract holds a reference to the model parameters in some external (decentralized) storage systems. In this case, the new model parameters cannot be calculated directly on the smart contract, as the smart contracts have limited functionality and are not able to download external information during execution. The most common approach is to have a set of devices performing the aggregation in parallel and submitting their final version. Through the smart contract, the majority of the devices must agree on what is the next global version. Whether these devices are the servers or the clients, it all depends on the architecture of the system.

% \begin{table}[!ht]
% \begin{tabular}{c|c} \hline \hline
%                   & Paper \\ \hline \hline
% On-chain Storage  & \cite{9274451, baffle, demo, 8733825, 9524833, 8894364, 9184854, 8893114}\\ \hline
% Off-chain Storage & \cite{10.1145/3319535.3363256, 10.48550/arxiv.2011.07516, 8945913, 10.48550/arxiv.2202.02817, 10.48550/arxiv.2007.03856, 10.48550/arxiv.1910.12603, Peyvandi2022, 9170559} \\      \hline
% \end{tabular}
% \caption{Model Parameter Storage}\label{tab:storage}
% \end{table}

Even though most implementations prefer an on-chain storage, these implementations also use custom blockchain implementations \cite{8733825, 9524833, 8894364, 9184854, 8893114}, which means that they can implement a platform that has different restrictions on how much data a smart contract can handle. When it comes to using already existing blockchain platforms such as Ethereum, most implementations prefer off-chain storage using a system such as the InterPlanetary File System\cite{10.48550/arxiv.2007.03856, 8945913, Peyvandi2022, 9170559, 10.1145/3319535.3363256, 10.48550/arxiv.2011.07516}.

\section{Participants Selection Algorithms}\label{related_work:participants_selection}

Usually, only some clients are asked to submit an update in each round. The process of choosing the clients that participate in each round can vary and have different costs.

In most cases, the number of clients and which clients specifically participate are chosen \textit{randomly} \cite{Peyvandi2022, demo, 9293091}. In some other systems, clients are allowed to take initiative, operating in a \textit{first come, first served} basis \cite{9184854, FANG20221}. In these systems, a pre-defined number of required clients is set and once enough clients have submitted their updates, the aggregation takes place.

\section{Scoring and Aggregation Algorithms}\label{related_work:scoring_techniques}

During the training process, each client produces its model parameter updates and communicates them to the servers through the blockchain. These parameters are then aggregated. However, there are different security aspects that should be taken into account here as the parameter updates creates a possibility for performing different attacks such as poisoning attacks and plagiarism attacks.

\begin{itemize}
    \item \textit{Poisoning attacks} happen when clients willingly send parameter updates that decrease the quality of the model. They may have been generated using an honestly unreliable data set, or done on purpose. To avoid other participants to provide unreliable data to degrade the model performance, there are dynamic verification techniques that allow to ignore low quality data \cite{10.48550/arxiv.2110.02182, 10.48550/arxiv.2104.10501}.
    
    \item \textit{Plagiarism attacks} happen when lazy clients plagiarize other client's models updates without really training their models, which can be addressed via pseudo-noise algorithms \cite{9403374}. In addition, plagiarism attacks within the same round can be avoided by secure communication methods, such as differential privacy \cite{10.48550/arxiv.2009.09338}, through which plagiarism attacks where a client reuses parameters from a previous round can be avoided by simply comparing the different client's updates.
\end{itemize}

Different solutions can be found in the literature regarding verification, mostly related to scoring algorithms:

\begin{itemize}
    \item \textit{Integrated Consensus Algorithms}. Some authors implement their own blockchain systems, which allows them to design their own consensus algorithm dedicated to FL. This consensus algorithm can use properties from the training process directly in order to detect which updates to accept or reject \cite{9293091, 10.1007/978-981-15-9213-3_12}.
    
    \item \textit{Random Committee}. In some works, a random set of participants is selected as committee and they must vote whether or not the model updates should be accepted. The decision to accept is based on the amount of votes that each submission received \cite{9159643}. There are different algorithms to decide on how voting works, such as voting in favor or against depending if it increases the performance relatively to the previous model or not.
    
    \item \textit{Scores-based System}. Scores-based system, also known as reputation-based system, usually work by giving clients with consistently high quality data and updates higher amounts of points. Then, the updates with less points are either rejected, or they have a smaller influence on the aggregation  \cite{10.48550/arxiv.2011.07516, 9170559, Peyvandi2022, 9292450}.
    
    \item The authors of \cite{8945913} implement a verification algorithm based on the trend of the validation error accuracy. To implement it, the model updates of each client are validated using a public validation data set known to both servers and clients. The result of this validation will also influence the reward distribution.

\end{itemize}

The costs of model update scoring have not been considered in most of the published papers and it is important to understand the trade-off between system costs and the different scoring algorithms\cite{9403374, 10.48550/arxiv.2110.02182}. It is also important to notice that different scoring algorithms directly affect the security model of the system \cite{10.48550/arxiv.2110.02182}.

\section{Privacy Mechanisms}\label{related_work:privacy}

Even though FL is already more secure than centralized ML in the sense that the raw data is never shared, the model weights are still a vector for attacks such as inference attacks \cite{10.1145/3298981}. Inference attacks are attacks, in which the weights are used to reverse-engineer the original data.

Additionally, in BFS systems, the weights are visible to all other client or participant since the blockchain provides an immutable, traceable and auditable record of the whole process. Consequently, it is important to reduce the vector of attacks when it comes to the model parameters. To combat this, authors use different privacy mechanisms, such as Differential Privacy \cite{10.48550/arxiv.2007.03856, Peyvandi2022, 9170559} and Homomorphic Encryption \cite{8945913, 8894364}.

\section{Conclusions}\label{related_work:conclusions}

From what mentioned above, we can draw the following conclusions. Firstly, there is a clear lack on how different components of a BFL system, such as consensus algorithms, scoring algorithms, number of clients, impact the accuracy, communication and computation costs of the system. This was also pointed out on a recent survey \cite{9403374} and it seems it was not addressed yet. Therefore, this work intends to fill in this gap by providing a detailed analysis on how some of these algorithms impact execution time, convergence, accuracy, communication and computation costs of the system.

Secondly, even though there are many works on designing BFL frameworks, very few are released to the public, or modular. We, therefore, will work on designing and implementing a modular BFL framework that can be easily changed to support new algorithms and will be available to the public to empower future research.

Lastly, to the best of our knowledge, there is only one work that argues that it is possible to implement Vertical Federated Learning in a BFS setting \cite{10.48550/arxiv.1912.04859}. However, it provides no implementation or design details. With that being said, we also would like to understand if such thing is feasible, and if so, implement it on our modular framework.

\begin{landscape}

\begin{table}

\begin{tabular}{c|c|c|c|c|c|c|c|c|c|c|c|c|c} \hline\hline
\multirow{2}{*}{Paper}    & \multicolumn{5}{|c|}{Platform}                       & \multicolumn{8}{|c}{Consensus}                           \\ \cline{2-14}
                          & Ethereum & Hyperledger & EOS & MultiChain & Custom & PoW & PoA & (p)BFT & PoS & PoFL & PoQ & Committee & Other \\ \hline\hline
\cite{10.48550/arxiv.2007.03856} & \checkmark        &             &     &            &        & \checkmark         &     &        &     &      &     &           &      \\ \hline
\cite{9233457}                   & \checkmark        &             &     &            &        & \checkmark         &     &        &     &      &     &           &      \\ \hline
\cite{9274451}                   & \checkmark        &             &     &            &        &           & \checkmark   &        &     &      &     &           &      \\ \hline
\cite{baffle}                    & \checkmark        &             &     &            &        &           & \checkmark   &        &     &      &     &           &      \\ \hline
\cite{9159643}                   & \checkmark        &             &     &            &        &           &     &        & \checkmark   &      &     &           &      \\ \hline
\cite{app8122663}                & \checkmark        &             &     & \checkmark          &        &           &     &        &     &      &     & \checkmark         &      \\ \hline
\cite{9006344}                   & \checkmark        &             &     &            &        &           &     &        &     &      &     &           &      \\\hline
\cite{Peyvandi2022}              & \checkmark        &             &     &            &        &           &     &        &     &      &     &           &      \\\hline
\cite{10.1145/3422337.3447837}   & \checkmark        & \checkmark           &     &            &        &           &     &        &     &      &     &           &      \\ \hline
\cite{10.1145/3319535.3363256}   & \checkmark        &             &     &            &        &           &     &        &     &      &     &           &      \\ \hline
\cite{10.48550/arxiv.1910.12603} & \checkmark        &             &     &            &        &           &     &        &     &      &     &           &      \\ \hline
\cite{10.48550/arxiv.2011.07516} & \checkmark        &             &     &            &        &           &     &        &     &      &     &           &      \\ \hline
\cite{demo}                      &          & \checkmark           &     &            &        &           & \checkmark   &        &     &      &     &           &      \\ \hline
\cite{10.48550/arxiv.2202.02817} &          & \checkmark           &     &            &        &           &     &        &     &      &     &           &      \\ \hline
\cite{8945913}                   &          &             & \checkmark   &            &        &           & \checkmark   &        &     &      &     &           &      \\ \hline
\cite{8733825}                   &          &             &     &            & \checkmark      & \checkmark         &     &        &     &      &     &           &      \\ \hline
\cite{10.48550/arxiv.2101.03300} &          &             &     &            & \checkmark      &           &     &        & \checkmark   &      &     &           &      \\ \hline
\cite{8998397}                   &          &             &     &            & \checkmark      &           &     &        & \checkmark   &      &     &           &      \\ \hline
\cite{8843900}                   &          &             &     &            & \checkmark      &           &     &        &     &      & \checkmark   &           &      \\ \hline
\cite{8894364}                   &          &             &     &            & \checkmark      &           &     &        &     &      &     & \checkmark         &      \\ \hline
\cite{9524833}                   &          &             &     &            & \checkmark      &           &     &        &     &      &     &           & \checkmark    \\ \hline
\cite{8905038}                   &          &             &     &            & \checkmark      &           &     &        &     &      &     &           & \checkmark    \\ \hline
\cite{FANG20221}                 &          &             &     &            & \checkmark      &           &     &        &     &      &     &           & \checkmark    \\ \hline
\end{tabular}

\caption{Blockchain Platforms and Consensus Algorithms}
\label{tab:platf_consensus}
\end{table}

\begin{table}
\ContinuedFloat
\begin{tabular}{c|c|c|c|c|c|c|c|c|c|c|c|c|c} \hline\hline
\multirow{2}{*}{Paper}    & \multicolumn{5}{|c|}{Platform}                       & \multicolumn{8}{|c}{Consensus}                           \\ \cline{2-14}
                          & Ethereum & Hyperledger & EOS & MultiChain & Custom & PoW & PoA & (p)BFT & PoS & PoFL & PoQ & Committee & Other \\ \hline\hline
\cite{9184854}                   &          &             &     &            & \checkmark      &           &     &        &     &      &     &           &       \\ \hline
\cite{8893114}                   &          &             &     &            & \checkmark      &           &     &        &     &      &     &           &       \\ \hline
\cite{10.48550/arxiv.2009.09338} &          &             &     &            & \checkmark      &           &     &        &     &      &     &           &       \\ \hline
\cite{8892848}                   &          &             &     &            & \checkmark      &           &     &        &     &      &     &           &       \\ \hline
\cite{10.48550/arxiv.2112.07938} &          &             &     &            &        & \checkmark         &     &        &     &      &     &           &       \\ \hline
\cite{9134967}                   &          &             &     &            &        & \checkmark         &     &        &     &      &     &           &       \\ \hline
\cite{10.48550/arxiv.1912.04859} &          &             &     &            &        & \checkmark         &     &        & \checkmark   &      &     &           &       \\ \hline
\cite{9079513}                   &          &             &     &            &        & \checkmark         &     & \checkmark      &     &      &     &           &       \\ \hline
\cite{9223754}                   &          &             &     &            &        & \checkmark         &     &        &     &      &     &           &       \\ \hline
\cite{9399813}                   &          &             &     &            &        & \checkmark         &     & \checkmark      & \checkmark   &      &     &           &       \\ \hline
\cite{9170559}                   &          &             &     &            &        &           &     & \checkmark      & \checkmark   &      &     &           &       \\ \hline
\cite{8994206}                   &          &             &     &            &        &           &     & \checkmark      &     &      &     &           &       \\ \hline
\cite{9311394}                   &          &             &     &            &        &           &     &        & \checkmark   &      &     &           &       \\ \hline
\cite{9170905}                   &          &             &     &            &        &           &     &        & \checkmark   &      &     &           &       \\ \hline
\cite{9347812}                   &          &             &     &            &        &           &     &        &     & \checkmark    &     &           &       \\ \hline
\cite{9127823}                   &          &             &     &            &        &           &     &        &     & \checkmark    &     &           &       \\ \hline
\cite{9292450}                   &          &             &     &            &        &           &     &        &     & \checkmark    &     &           &       \\ \hline
\cite{9321132}                   &          &             &     &            &        &           &     &        &     &      &     & \checkmark         &       \\ \hline
\cite{9210531}                   &          &             &     &            &        &           &     &        &     &      &     & \checkmark         &       \\ \hline
\cite{pirate}                    &          &             &     &            &        &           &     &        &     &      &     & \checkmark         &       \\ \hline
\cite{8832210}                   &          &             &     &            &        &           &     &        &     &      &     & \checkmark         &       \\ \hline
\cite{9293091}                   &          &             &     &            &        &           &     &        &     &      &     & \checkmark         &       \\ \hline
\end{tabular}

\caption{Blockchain Platforms and Consensus Algorithms (Continued)}
\end{table}

\end{landscape}

\begin{table}[ht]
\centering
\caption{Data Distribution, Data Partition and Datasets}
\label{tab:data_distribution}
\begin{tabular}{c|c|c|c}
\hline \hline
                                    & Data Distribution & Data Partition & Dataset                            \\ \hline \hline
\cite{10.1145/3319535.3363256}      & I                 & H              & Breast Cancer Dataset              \\ \hline
\cite{8905038}                      & N                 & H              &                                    \\ \hline
% \cite{10.48550/arxiv.2011.07516}    & --                & --             & MNIST                              \\ \hline
\cite{9524833}                      & N                 & --             & --                                 \\ \hline
\cite{9127823}                      & I                 & H              & MNIST, CIFAR10                     \\ \hline
\cite{10.48550/arxiv.2101.03300}    & --                & H              & MNIST                              \\ \hline
\cite{9159643}                      & I                 & H              & MovieLens                          \\ \hline
\cite{10.1145/3422337.3447837}      & I, N              & --             & --                                 \\ \hline
\cite{9223754}                      & I                 & H              & MNIST                              \\ \hline
\cite{FANG20221}                    & --                & H              & CIFAR10                            \\ \hline
\cite{9399813}                      & I                 & H              & MNIST                              \\ \hline
\cite{9184854}                      & --                & H              & --                                 \\ \hline
\cite{8851649}                      & I                 & H              & MNIST                              \\ \hline
\cite{8994206}                      & I                 & H              & MNIST                              \\ \hline
\cite{8733825}                      & I                 & H              & --                                 \\ \hline
\cite{8893114}                      & N                 & H              & MNIST                              \\ \hline
\cite{9274451}                      & N                 & H              & MNIST                              \\ \hline
\cite{9293091}                      & I                 & H              & FEMNIST                            \\ \hline
\cite{8843900}                      & I                 & H              & Reuters, 20News                    \\ \hline
\cite{8998397}                      & I                 & H              & Uber Pickups, MNIST                \\ \hline
\cite{9311394}                      & I                 & H              & MNIST                              \\ \hline
\cite{9170905}                      & I                 & H              & CIFAR10                            \\ \hline
\cite{10.48550/arxiv.2009.09338}    & --                & H              & --                                 \\ \hline
\cite{8892848}                      & I, N              & H              & --                                 \\ \hline
\cite{8945913}                      & --                & H              & MNIST                              \\ \hline
\cite{10.48550/arxiv.2202.02817}    & N                 & H              & MNIST, CIFAR10                     \\ \hline
\cite{10.48550/arxiv.2007.03856}    & --                & H              & --                                 \\ \hline
\cite{10.48550/arxiv.1912.04859}    & I, N              & H,V            & --                                 \\ \hline
\cite{10.48550/arxiv.1910.12603}    & I, N              & H              & --                                 \\ \hline
\cite{9321132}                      & I                 & H              & MNIST                              \\ \hline
\cite{Peyvandi2022}                 & N                 & H              & --                                 \\ \hline
\cite{9079513}                      & I                 & H              & --                                 \\ \hline
\cite{app8122663}                   & I                 & H              & CICIDS 2017                        \\ \hline
\cite{9347812}                      & I                 & H              & CIFAR10                            \\ \hline
\cite{9134967}                      & I, N              & H              & MNIST, CIFAR10                     \\ \hline
\cite{baffle}                       & I                 & H              & NYC 2018 Taxi                      \\ \hline
\cite{9210531}                      & I                 & H              & LFW, MNIST, CelebA, CASIA          \\ \hline
\cite{8894364}                      & --                & H              & MNIST                              \\ \hline
\cite{10.48550/arxiv.2112.07938}    & --                & H              & --                                 \\ \hline
\cite{demo}                         & I                 & H              & MNIST                              \\ \hline
\cite{9233457}                      & I                 & H              & Air-Conditioning                   \\ \hline
\cite{9170559}                      & I                 & H              & MNIST                              \\ \hline
\cite{pirate}                       & I                 & H              & --                                 \\ \hline
\end{tabular}
\caption*{I: IID, N: Non-IID, H: Horizontal, V: Vertical}
\end{table}
