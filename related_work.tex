In Blockchain-based Federated Learning (BFL) systems, there are different algorithms that need to be taken into consideration when building the system. In this chapter, we go over these algorithms and their variations used in the literature in order to find the gap that we will try to fill in this thesis.

\section{Consensus Algorithms}\label{related_work:consensus_algorithms}

As it can be seen from \autoref{tab:platf_consensus}, various consensus algorithms have been used for BFL systems. Below is a summary of each of these consensus algorithms. In most works, authors chose to use an already existing consensus algorithm, such as Proof of Work \cite{10.48550/arxiv.2007.03856, 9233457, 8733825, 10.48550/arxiv.2112.07938, 9134967, 10.48550/arxiv.1912.04859, 9079513, 9223754, 9399813}, Proof of Stake \cite{9159643, 10.48550/arxiv.2101.03300, 8998397, 10.48550/arxiv.1912.04859, 9399813, 9170559, 9311394, 9170905} and Proof of Authority \cite{9274451, baffle, demo, 8945913}. In addition, in the majority of the existing works that used an already existing consensus algorithm, the BFL system is built on top of an already existing blockchain platform.

There are works such as \cite{9347812, 10.48550/arxiv.2007.15145, 8843900} that integrate the consensus algorithm of the blockchain with the model training process in order to preserve energy and resource consumption \cite{9293091}. However, they are either not publicly available or cannot be easily applied to already existing blockchain platforms because they require internal changes.

Finally, even though there are some works that analyze the resource and energy consumption of blockchain consensus algorithm in the context of BFL systems, there is a lack of analysis for already existing consensus algorithms. This is also mentioned by survey papers such as \cite{9403374, 10.48550/arxiv.2112.07938}. Therefore, we intend to fill in the gap by providing an impact analysis of different consensus algorithms on already existing blockchain platforms.

\section{Model Parameter Storage}\label{related_work:param_storage}

Another important component of BFL systems is the location, where the model parameters are stored in order to be shared with the servers. According to the literature, the model parameters may either be stored on-chain, i.e., in the blockchain itself, or off-chain, i.e., in a separate storage provider \cite{10.48550/arxiv.2104.10501}.

With the \textit{on-chain storage} \cite{9274451, baffle, demo, 8733825, 9524833, 8894364, 9184854, 8893114}, the smart contract stores the model parameters itself, which means that the parameters themselves will be stored in the blockchain. However, most blockchain platforms have a limit on how large a block can be and, consequently, the amount of data that can be stored per contract is limited \cite{9274451}. In these cases, smart contracts are chunked, i.e., a single contract is split into many different contracts that hold smaller chunks of the parameters \cite{baffle}. In addition, this allows for the new model parameters to be directly calculated through the smart contract as the values are directly accessible  \cite{9274451}.
    
With the \textit{off-chain storage} \cite{10.1145/3319535.3363256, 10.48550/arxiv.2011.07516, 8945913, 10.48550/arxiv.2202.02817, 10.48550/arxiv.2007.03856, 10.48550/arxiv.1910.12603, Peyvandi2022, 9170559}, the smart contract holds a reference to the model parameters in some external (decentralized) storage systems. In this case, the new model parameters cannot be calculated directly on the smart contract, as the smart contracts have limited functionality and are not able to download external information during execution. Instead, a set of devices perform the aggregation in parallel and submit their aggregation. Through the smart contract, the majority of the devices must agree on what is the next global aggregation. Whether these devices are the servers or the clients, it all depends on the architecture of the system.

Even though most implementations prefer an on-chain storage, these implementations also use custom blockchain implementations \cite{8733825, 9524833, 8894364, 9184854, 8893114}, which means that they can implement a platform that has different restrictions on how much data a smart contract can handle. When it comes to using already existing blockchain platforms such as Ethereum, most implementations prefer off-chain storage using a system such as the InterPlanetary File System\cite{10.48550/arxiv.2007.03856, 8945913, Peyvandi2022, 9170559, 10.1145/3319535.3363256, 10.48550/arxiv.2011.07516}.

\section{Participants Selection Algorithms}\label{related_work:participants_selection}

Usually, only some clients are asked to submit a model update in each round. The process of choosing the clients that participate in each round can vary and have different costs. In most works, such as \cite{Peyvandi2022, demo, 9293091}, the number of clients and which clients specifically participate are chosen \textit{randomly}. In other systems such as \cite{9184854, FANG20221}, clients are allowed to take initiative, operating in a \textit{first come, first served} basis. The survey \cite{9403374} mentions that there is a lack of analysis on how the selection of the participants impacts the accuracy of the BFL system, as well as the communication and computation costs.

\section{Scoring and Aggregation Algorithms}\label{related_work:scoring_techniques}

Different solutions can be found in the literature regarding verification techniques. The authors of \cite{9159643} created a committee-based verification mechanism. To implement it, they deploy a verification smart contract on the blockchain, which periodically elects different clients as committee members. Then, the committee is responsible for voting if the submitted model updates are valid or not.

The authors of \cite{8945913} implement a verification algorithm based on the trend of the validation error accuracy. To implement it, the model updates of each client are validated using a public validation data set known to both servers and clients. The result of this validation also influences the reward distribution.

Scores-based systems \cite{10.48550/arxiv.2011.07516, 9170559, Peyvandi2022, 9292450}, also known as reputation-based systems, are, by far, the most common among the literature and also the only ones that can be applied to already existing blockchain platforms. These work by giving clients with consistently high quality data and updates higher amounts of points. Then, the updates with less points are either rejected, or they have a smaller influence on the aggregation.

In most of these works, as also noted by \cite{9403374, 10.48550/arxiv.2110.02182}, the costs of scoring algorithms have not been considered. It is important to understand the trade-off between communication and computation costs, and the usage of the different scoring algorithms. The authors of \cite{10.48550/arxiv.2110.02182} also mention that there is a lack of comparison of the different security models of the system when using different scoring algorithms.

\section{Privacy Mechanisms}\label{related_work:privacy}

The majority of the reviewed works did not mention which privacy mechanism they used. However, we found Differential Privacy \cite{10.48550/arxiv.2007.03856, Peyvandi2022, 9170559} to be the most common mechanism, followed by the Homomorphic Encryption \cite{8945913, 8894364}. In addition, the authors of \cite{9403374} also point out the lack of consideration for how privacy mechanisms may impact the system, both in terms of accuracy, as well as resource consumption. From our review, we can confirm that this is still the case.

\section{Other Remarks}\label{related_work:other_remarks}

In addition, there are several works designing BFL frameworks \cite{10.1145/3422337.3447837, 8945913, 10.48550/arxiv.1910.12603, 10.48550/arxiv.2110.02182}, but very few of them provide the source code or build a framework that is intended to be used by others.sign Providing the source code is extremely important when it comes to reproducibility and verification, so that others can analyze it and do further research using it.

Finally, as it can be seen in \autoref{tab:data_distribution}, only \cite{10.48550/arxiv.1912.04859} mentions that they implement Vertical Federated Learning in a BFS setting. However, it provides no implementation or design details.

\section{Conclusions}\label{related_work:conclusions}

From the literature review, we can draw the following conclusions. Firstly, there is a clear lack on how different components of a BFL system, such as consensus algorithms, scoring algorithms, number of clients, impact the accuracy, communication and computation costs of the system. Consequently, this work intends to fill in this gap by providing a detailed analysis on how some of these algorithms impact execution time, convergence, accuracy, communication and computation costs of the system.

Secondly, even though there are many works on designing BFL frameworks, very few are released to the public, or modular. We, therefore, will work on designing and implementing a modular BFL framework that can be easily changed to support new algorithms and will be available to the public to empower future research.

Lastly, to the best of our knowledge, there is only one work that argues that it is possible to implement Vertical Federated Learning in a BFS setting \cite{10.48550/arxiv.1912.04859}. However, it provides no implementation or design details. With that being said, we also would like to understand if such thing is feasible, and if so, implement it on our modular framework.

\begin{landscape}

\begin{table}

\begin{tabular}{c|c|c|c|c|c|c|c|c|c|c|c|c|c} \hline\hline
\multirow{2}{*}{Paper}    & \multicolumn{5}{|c|}{Platform}                       & \multicolumn{8}{|c}{Consensus}                           \\ \cline{2-14}
                          & Ethereum & Hyperledger & EOS & MultiChain & Custom & PoW & PoA & (p)BFT & PoS & PoFL & PoQ & Committee & Other \\ \hline\hline
\cite{10.48550/arxiv.2007.03856} & \checkmark        &             &     &            &        & \checkmark         &     &        &     &      &     &           &      \\ \hline
\cite{9233457}                   & \checkmark        &             &     &            &        & \checkmark         &     &        &     &      &     &           &      \\ \hline
\cite{9274451}                   & \checkmark        &             &     &            &        &           & \checkmark   &        &     &      &     &           &      \\ \hline
\cite{baffle}                    & \checkmark        &             &     &            &        &           & \checkmark   &        &     &      &     &           &      \\ \hline
\cite{9159643}                   & \checkmark        &             &     &            &        &           &     &        & \checkmark   &      &     &           &      \\ \hline
\cite{app8122663}                & \checkmark        &             &     & \checkmark          &        &           &     &        &     &      &     & \checkmark         &      \\ \hline
\cite{9006344}                   & \checkmark        &             &     &            &        &           &     &        &     &      &     &           &      \\\hline
\cite{Peyvandi2022}              & \checkmark        &             &     &            &        &           &     &        &     &      &     &           &      \\\hline
\cite{10.1145/3422337.3447837}   & \checkmark        & \checkmark           &     &            &        &           &     &        &     &      &     &           &      \\ \hline
\cite{10.1145/3319535.3363256}   & \checkmark        &             &     &            &        &           &     &        &     &      &     &           &      \\ \hline
\cite{10.48550/arxiv.1910.12603} & \checkmark        &             &     &            &        &           &     &        &     &      &     &           &      \\ \hline
\cite{10.48550/arxiv.2011.07516} & \checkmark        &             &     &            &        &           &     &        &     &      &     &           &      \\ \hline
\cite{demo}                      &          & \checkmark           &     &            &        &           & \checkmark   &        &     &      &     &           &      \\ \hline
\cite{10.48550/arxiv.2202.02817} &          & \checkmark           &     &            &        &           &     &        &     &      &     &           &      \\ \hline
\cite{8945913}                   &          &             & \checkmark   &            &        &           & \checkmark   &        &     &      &     &           &      \\ \hline
\cite{8733825}                   &          &             &     &            & \checkmark      & \checkmark         &     &        &     &      &     &           &      \\ \hline
\cite{10.48550/arxiv.2101.03300} &          &             &     &            & \checkmark      &           &     &        & \checkmark   &      &     &           &      \\ \hline
\cite{8998397}                   &          &             &     &            & \checkmark      &           &     &        & \checkmark   &      &     &           &      \\ \hline
\cite{8843900}                   &          &             &     &            & \checkmark      &           &     &        &     &      & \checkmark   &           &      \\ \hline
\cite{8894364}                   &          &             &     &            & \checkmark      &           &     &        &     &      &     & \checkmark         &      \\ \hline
\cite{9524833}                   &          &             &     &            & \checkmark      &           &     &        &     &      &     &           & \checkmark    \\ \hline
\cite{8905038}                   &          &             &     &            & \checkmark      &           &     &        &     &      &     &           & \checkmark    \\ \hline
\cite{FANG20221}                 &          &             &     &            & \checkmark      &           &     &        &     &      &     &           & \checkmark    \\ \hline
\end{tabular}

\caption{Blockchain Platforms and Consensus Algorithms}
\label{tab:platf_consensus}
\end{table}

\begin{table}
\ContinuedFloat
\begin{tabular}{c|c|c|c|c|c|c|c|c|c|c|c|c|c} \hline\hline
\multirow{2}{*}{Paper}    & \multicolumn{5}{|c|}{Platform}                       & \multicolumn{8}{|c}{Consensus}                           \\ \cline{2-14}
                          & Ethereum & Hyperledger & EOS & MultiChain & Custom & PoW & PoA & (p)BFT & PoS & PoFL & PoQ & Committee & Other \\ \hline\hline
\cite{9184854}                   &          &             &     &            & \checkmark      &           &     &        &     &      &     &           &       \\ \hline
\cite{8893114}                   &          &             &     &            & \checkmark      &           &     &        &     &      &     &           &       \\ \hline
\cite{10.48550/arxiv.2009.09338} &          &             &     &            & \checkmark      &           &     &        &     &      &     &           &       \\ \hline
\cite{8892848}                   &          &             &     &            & \checkmark      &           &     &        &     &      &     &           &       \\ \hline
\cite{10.48550/arxiv.2112.07938} &          &             &     &            &        & \checkmark         &     &        &     &      &     &           &       \\ \hline
\cite{9134967}                   &          &             &     &            &        & \checkmark         &     &        &     &      &     &           &       \\ \hline
\cite{10.48550/arxiv.1912.04859} &          &             &     &            &        & \checkmark         &     &        & \checkmark   &      &     &           &       \\ \hline
\cite{9079513}                   &          &             &     &            &        & \checkmark         &     & \checkmark      &     &      &     &           &       \\ \hline
\cite{9223754}                   &          &             &     &            &        & \checkmark         &     &        &     &      &     &           &       \\ \hline
\cite{9399813}                   &          &             &     &            &        & \checkmark         &     & \checkmark      & \checkmark   &      &     &           &       \\ \hline
\cite{9170559}                   &          &             &     &            &        &           &     & \checkmark      & \checkmark   &      &     &           &       \\ \hline
\cite{8994206}                   &          &             &     &            &        &           &     & \checkmark      &     &      &     &           &       \\ \hline
\cite{9311394}                   &          &             &     &            &        &           &     &        & \checkmark   &      &     &           &       \\ \hline
\cite{9170905}                   &          &             &     &            &        &           &     &        & \checkmark   &      &     &           &       \\ \hline
\cite{9347812}                   &          &             &     &            &        &           &     &        &     & \checkmark    &     &           &       \\ \hline
\cite{9127823}                   &          &             &     &            &        &           &     &        &     & \checkmark    &     &           &       \\ \hline
\cite{9292450}                   &          &             &     &            &        &           &     &        &     & \checkmark    &     &           &       \\ \hline
\cite{9321132}                   &          &             &     &            &        &           &     &        &     &      &     & \checkmark         &       \\ \hline
\cite{9210531}                   &          &             &     &            &        &           &     &        &     &      &     & \checkmark         &       \\ \hline
\cite{pirate}                    &          &             &     &            &        &           &     &        &     &      &     & \checkmark         &       \\ \hline
\cite{8832210}                   &          &             &     &            &        &           &     &        &     &      &     & \checkmark         &       \\ \hline
\cite{9293091}                   &          &             &     &            &        &           &     &        &     &      &     & \checkmark         &       \\ \hline
\end{tabular}

\caption{Blockchain Platforms and Consensus Algorithms (Continued)}
\end{table}

\end{landscape}

\begin{table}[ht]
\centering
\caption{Data Distribution, Data Partition and Datasets}
\label{tab:data_distribution}
\begin{tabular}{c|c|c|c}
\hline \hline
                                    & Data Distribution & Data Partition & Dataset                            \\ \hline \hline
\cite{10.1145/3319535.3363256}      & I                 & H              & Breast Cancer Dataset              \\ \hline
\cite{8905038}                      & N                 & H              &                                    \\ \hline
% \cite{10.48550/arxiv.2011.07516}    & --                & --             & MNIST                              \\ \hline
\cite{9524833}                      & N                 & --             & --                                 \\ \hline
\cite{9127823}                      & I                 & H              & MNIST, CIFAR10                     \\ \hline
\cite{10.48550/arxiv.2101.03300}    & --                & H              & MNIST                              \\ \hline
\cite{9159643}                      & I                 & H              & MovieLens                          \\ \hline
\cite{10.1145/3422337.3447837}      & I, N              & --             & --                                 \\ \hline
\cite{9223754}                      & I                 & H              & MNIST                              \\ \hline
\cite{FANG20221}                    & --                & H              & CIFAR10                            \\ \hline
\cite{9399813}                      & I                 & H              & MNIST                              \\ \hline
\cite{9184854}                      & --                & H              & --                                 \\ \hline
\cite{8851649}                      & I                 & H              & MNIST                              \\ \hline
\cite{8994206}                      & I                 & H              & MNIST                              \\ \hline
\cite{8733825}                      & I                 & H              & --                                 \\ \hline
\cite{8893114}                      & N                 & H              & MNIST                              \\ \hline
\cite{9274451}                      & N                 & H              & MNIST                              \\ \hline
\cite{9293091}                      & I                 & H              & FEMNIST                            \\ \hline
\cite{8843900}                      & I                 & H              & Reuters, 20News                    \\ \hline
\cite{8998397}                      & I                 & H              & Uber Pickups, MNIST                \\ \hline
\cite{9311394}                      & I                 & H              & MNIST                              \\ \hline
\cite{9170905}                      & I                 & H              & CIFAR10                            \\ \hline
\cite{10.48550/arxiv.2009.09338}    & --                & H              & --                                 \\ \hline
\cite{8892848}                      & I, N              & H              & --                                 \\ \hline
\cite{8945913}                      & --                & H              & MNIST                              \\ \hline
\cite{10.48550/arxiv.2202.02817}    & N                 & H              & MNIST, CIFAR10                     \\ \hline
\cite{10.48550/arxiv.2007.03856}    & --                & H              & --                                 \\ \hline
\cite{10.48550/arxiv.1912.04859}    & I, N              & H,V            & --                                 \\ \hline
\cite{10.48550/arxiv.1910.12603}    & I, N              & H              & --                                 \\ \hline
\cite{9321132}                      & I                 & H              & MNIST                              \\ \hline
\cite{Peyvandi2022}                 & N                 & H              & --                                 \\ \hline
\cite{9079513}                      & I                 & H              & --                                 \\ \hline
\cite{app8122663}                   & I                 & H              & CICIDS 2017                        \\ \hline
\cite{9347812}                      & I                 & H              & CIFAR10                            \\ \hline
\cite{9134967}                      & I, N              & H              & MNIST, CIFAR10                     \\ \hline
\cite{baffle}                       & I                 & H              & NYC 2018 Taxi                      \\ \hline
\cite{9210531}                      & I                 & H              & LFW, MNIST, CelebA, CASIA          \\ \hline
\cite{8894364}                      & --                & H              & MNIST                              \\ \hline
\cite{10.48550/arxiv.2112.07938}    & --                & H              & --                                 \\ \hline
\cite{demo}                         & I                 & H              & MNIST                              \\ \hline
\cite{9233457}                      & I                 & H              & Air-Conditioning                   \\ \hline
\cite{9170559}                      & I                 & H              & MNIST                              \\ \hline
\cite{pirate}                       & I                 & H              & --                                 \\ \hline
\end{tabular}
\caption*{I: IID, N: Non-IID, H: Horizontal, V: Vertical}
\end{table}
