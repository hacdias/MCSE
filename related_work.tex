\section{Federated Learning Optimization Algorithms}

\section{Blockchain-enabled Federated Learning}

% \section{Blockchain Consensus Mechanism}

% \begin{table}[!ht]
\centering
\caption{Consensus Mechanisms}
\label{tab:consensus_mechanisms}
\begin{tabular}{c|c|c|c|c|c|c|c|c}
\hline \hline
                                    & PoW           & PoA           & PoS           & PoFL          & PoQ           & (p)BFT        & Committee     & Other         \\ \hline \hline
\cite{8905038}                      &               &               &               &               &               &               &               & \checkmark    \\ \hline
\cite{9524833}                      &               &               &               &               &               &               &               & \checkmark    \\ \hline
\cite{9127823}                      &               &               &               & \checkmark    &               &               &               &               \\ \hline
\cite{10.48550/arxiv.2101.03300}    &               &               & \checkmark    &               &               &               &               &               \\ \hline
\cite{9159643}                      &               &               & \checkmark    &               &               &               &               &               \\ \hline
\cite{9223754}                      & \checkmark    &               &               &               &               &               &               &               \\ \hline
\cite{FANG20221}                    &               &               &               &               &               &               &               & \checkmark    \\ \hline
\cite{9399813}                      & \checkmark    &               & \checkmark    &               &               & \checkmark    &               &               \\ \hline
\cite{8832210}                      &               &               &               &               &               &               & \checkmark    &               \\ \hline
\cite{8994206}                      &               &               &               &               &               & \checkmark    &               &               \\ \hline
\cite{8733825}                      & \checkmark    &               &               &               &               &               &               &               \\ \hline
\cite{9274451}                      &               & \checkmark    &               &               &               &               &               &               \\ \hline
\cite{9293091}                      &               &               &               &               &               &               & \checkmark    &               \\ \hline 
\cite{8843900}                      &               &               &               &               & \checkmark    &               &               &               \\ \hline
\cite{8998397}                      &               &               & \checkmark    &               &               &               &               &               \\ \hline
\cite{9311394}                      &               &               & \checkmark    &               &               &               &               &               \\ \hline
\cite{9170905}                      &               &               & \checkmark    &               &               &               &               &               \\ \hline
\cite{8945913}                      &               & \checkmark    &               &               &               &               &               &               \\ \hline
\cite{10.48550/arxiv.2007.03856}    & \checkmark    &               &               &               &               &               &               &               \\ \hline
\cite{10.48550/arxiv.1912.04859}    & \checkmark    &               & \checkmark    &               &               &               &               &               \\ \hline
\cite{9321132}                      &               &               &               &               &               &               & \checkmark    &               \\ \hline
\cite{9079513}                      & \checkmark    &               &               &               &               & \checkmark    &               &               \\ \hline
\cite{app8122663}                   &               &               &               &               &               &               & \checkmark    &               \\ \hline
\cite{9347812}                      &               &               &               & \checkmark    &               &               &               &               \\ \hline
\cite{9134967}                      & \checkmark    &               &               &               &               &               &               &               \\ \hline
\cite{baffle}                       &               & \checkmark    &               &               &               &               &               &               \\ \hline
\cite{9292450}                      &               &               &               & \checkmark    &               &               &               &               \\ \hline
\cite{9210531}                      &               &               &               &               &               &               & \checkmark    &               \\ \hline
\cite{8894364}                      &               &               &               &               &               &               & \checkmark    &               \\ \hline
\cite{10.48550/arxiv.2112.07938}    & \checkmark    &               &               &               &               &               &               &               \\ \hline
\cite{demo}                         &               & \checkmark    &               &               &               &               &               &               \\ \hline
\cite{9233457}                      & \checkmark    &               &               &               &               &               &               &               \\ \hline
\cite{9170559}                      &               &               & \checkmark    &               &               & \checkmark    &               &               \\ \hline
\cite{pirate}                       &               &               &               &               &               &               & \checkmark    &               \\ \hline
\end{tabular}
\end{table}


% \subsection{Conclusions}

% \section{Blockchain Platforms}

% 

% \subsection{Conclusions}

% \section{Model Parameter Storage}

% \subsection{Conclusion}

% \section{Participants Selection Mechanism}

% \subsection{Conclusions}

% \section{Cross Verification Techniques}

% \subsection{Conclusions}

% \section{Data Distribution, Partition and Dataset}

% \begin{table}[ht]
\centering
\caption{Data Distribution, Data Partition and Datasets}
\label{tab:data_distribution}
\begin{tabular}{c|c|c|c}
\hline \hline
                                    & Data Distribution & Data Partition & Dataset                            \\ \hline \hline
\cite{10.1145/3319535.3363256}      & I                 & H              & Breast Cancer Dataset              \\ \hline
\cite{8905038}                      & N                 & H              &                                    \\ \hline
% \cite{10.48550/arxiv.2011.07516}    & --                & --             & MNIST                              \\ \hline
\cite{9524833}                      & N                 & --             & --                                 \\ \hline
\cite{9127823}                      & I                 & H              & MNIST, CIFAR10                     \\ \hline
\cite{10.48550/arxiv.2101.03300}    & --                & H              & MNIST                              \\ \hline
\cite{9159643}                      & I                 & H              & MovieLens                          \\ \hline
\cite{10.1145/3422337.3447837}      & I, N              & --             & --                                 \\ \hline
\cite{9223754}                      & I                 & H              & MNIST                              \\ \hline
\cite{FANG20221}                    & --                & H              & CIFAR10                            \\ \hline
\cite{9399813}                      & I                 & H              & MNIST                              \\ \hline
\cite{9184854}                      & --                & H              & --                                 \\ \hline
\cite{8851649}                      & I                 & H              & MNIST                              \\ \hline
\cite{8994206}                      & I                 & H              & MNIST                              \\ \hline
\cite{8733825}                      & I                 & H              & --                                 \\ \hline
\cite{8893114}                      & N                 & H              & MNIST                              \\ \hline
\cite{9274451}                      & N                 & H              & MNIST                              \\ \hline
\cite{9293091}                      & I                 & H              & FEMNIST                            \\ \hline
\cite{8843900}                      & I                 & H              & Reuters, 20News                    \\ \hline
\cite{8998397}                      & I                 & H              & Uber Pickups, MNIST                \\ \hline
\cite{9311394}                      & I                 & H              & MNIST                              \\ \hline
\cite{9170905}                      & I                 & H              & CIFAR10                            \\ \hline
\cite{10.48550/arxiv.2009.09338}    & --                & H              & --                                 \\ \hline
\cite{8892848}                      & I, N              & H              & --                                 \\ \hline
\cite{8945913}                      & --                & H              & MNIST                              \\ \hline
\cite{10.48550/arxiv.2202.02817}    & N                 & H              & MNIST, CIFAR10                     \\ \hline
\cite{10.48550/arxiv.2007.03856}    & --                & H              & --                                 \\ \hline
\cite{10.48550/arxiv.1912.04859}    & I, N              & H,V            & --                                 \\ \hline
\cite{10.48550/arxiv.1910.12603}    & I, N              & H              & --                                 \\ \hline
\cite{9321132}                      & I                 & H              & MNIST                              \\ \hline
\cite{Peyvandi2022}                 & N                 & H              & --                                 \\ \hline
\cite{9079513}                      & I                 & H              & --                                 \\ \hline
\cite{app8122663}                   & I                 & H              & CICIDS 2017                        \\ \hline
\cite{9347812}                      & I                 & H              & CIFAR10                            \\ \hline
\cite{9134967}                      & I, N              & H              & MNIST, CIFAR10                     \\ \hline
\cite{baffle}                       & I                 & H              & NYC 2018 Taxi                      \\ \hline
\cite{9210531}                      & I                 & H              & LFW, MNIST, CelebA, CASIA          \\ \hline
\cite{8894364}                      & --                & H              & MNIST                              \\ \hline
\cite{10.48550/arxiv.2112.07938}    & --                & H              & --                                 \\ \hline
\cite{demo}                         & I                 & H              & MNIST                              \\ \hline
\cite{9233457}                      & I                 & H              & Air-Conditioning                   \\ \hline
\cite{9170559}                      & I                 & H              & MNIST                              \\ \hline
\cite{pirate}                       & I                 & H              & --                                 \\ \hline
\end{tabular}
\caption*{I: IID, N: Non-IID, H: Horizontal, V: Vertical}
\end{table}

% \subsection{Conclusions}

% \section{Sychronous vs Asynchronous Rounds}

% \subsection{Conclusions}

% \section{Communication Cost}

% \subsection{Conclusions}


\section{Blockchain-specific Aspects}

In a BFL system, there are two different systems at play: the blockchain and the machine learning environment. They are both connected, yet they are separate entities. In this section, we analyze blockchain-specific aspects regarding BFL implementations such as blockchain platforms and consensus algorithms.

\subsection{Blockchain Platforms}

As explained in \autoref{preliminaries:blockchain}, blockchain platforms allow developers to build applications on top of blockchain technologies. Even though all platforms are based on the concept of blockchain, they all have different characteristics and restrictions, as well as different sets of features.

As can be seen in \autoref{tab:blockchain_platforms}, the majority of papers analyzed chose an already existing platform. Among those, Ethereum was the most popular. In addition, it is worth mentioning that a large amount of authors also implemented their own blockchain platform. By implementing their own platform, it is easier to overcome certain restrictions such as limits on data per block \cite{8733825, 9524833}.

\begin{table}[!ht]
\centering
\caption{Blockchain Platforms}
\label{tab:blockchain_platforms}
\begin{tabular}{c|c|c|c|c|c}
\hline \hline
                                    & Ethereum      & Hyperledger   & EOS           & MultiChain    & Custom        \\ \hline \hline
\cite{10.1145/3319535.3363256}      & \checkmark    &               &               &               &               \\ \hline
\cite{8905038}                      &               &               &               &               & \checkmark    \\ \hline
\cite{10.48550/arxiv.2011.07516}    & \checkmark    &               &               &               &               \\ \hline
\cite{9524833}                      &               &               &               &               & \checkmark    \\ \hline
\cite{10.48550/arxiv.2101.03300}    &               &               &               &               & \checkmark    \\ \hline
\cite{9159643}                      & \checkmark    &               &               &               &               \\ \hline
\cite{10.1145/3422337.3447837}      & \checkmark    & \checkmark    &               &               &               \\ \hline
\cite{FANG20221}                    &               &               &               &               & \checkmark    \\ \hline
\cite{9184854}                      &               &               &               &               & \checkmark    \\ \hline
\cite{8733825}                      &               &               &               &               & \checkmark    \\ \hline
\cite{8893114}                      &               &               &               &               & \checkmark    \\ \hline
\cite{9274451}                      & \checkmark    &               &               &               &               \\ \hline
\cite{8843900}                      &               &               &               &               & \checkmark    \\ \hline
\cite{8998397}                      &               &               &               &               & \checkmark    \\ \hline
\cite{10.48550/arxiv.2009.09338}    &               &               &               &               & \checkmark    \\ \hline
\cite{8892848}                      &               &               &               &               & \checkmark    \\ \hline
\cite{8945913}                      &               &               & \checkmark    &               &               \\ \hline
\cite{10.48550/arxiv.2202.02817}    &               & \checkmark    &               &               &               \\ \hline
\cite{10.48550/arxiv.2007.03856}    & \checkmark    &               &               &               &               \\ \hline
\cite{10.48550/arxiv.1910.12603}    & \checkmark    &               &               &               &               \\ \hline
\cite{Peyvandi2022}                 & \checkmark    &               &               &               &               \\ \hline
\cite{app8122663}                   & \checkmark    &               &               & \checkmark    &               \\ \hline
\cite{baffle}                       & \checkmark    &               &               &               &               \\ \hline
\cite{9006344}                      & \checkmark    &               &               &               &               \\ \hline
\cite{8894364}                      &               &               &               &               & \checkmark    \\ \hline
\cite{demo}                         &               & \checkmark    &               &               &               \\ \hline
\cite{9233457}                      & \checkmark    &               &               &               &               \\ \hline
\end{tabular}
\end{table}

\subsection{Consensus Algorithms}

One of the most important aspects of blockchain technology is the consensus algorithm. Consensus is the process of reaching an agreement on a single value among different distributed processes. These algorithms are designed to be reliable even on networks that have unreliable nodes. In blockchain, the consensus algorithm is used to reach consensus on the next block of the chain.

\begin{table}[!ht]
\centering
\caption{Consensus Mechanisms}
\label{tab:consensus_mechanisms}
\begin{tabular}{c|c|c|c|c|c|c|c|c}
\hline \hline
                                    & PoW           & PoA           & PoS           & PoFL          & PoQ           & (p)BFT        & Committee     & Other         \\ \hline \hline
\cite{8905038}                      &               &               &               &               &               &               &               & \checkmark    \\ \hline
\cite{9524833}                      &               &               &               &               &               &               &               & \checkmark    \\ \hline
\cite{9127823}                      &               &               &               & \checkmark    &               &               &               &               \\ \hline
\cite{10.48550/arxiv.2101.03300}    &               &               & \checkmark    &               &               &               &               &               \\ \hline
\cite{9159643}                      &               &               & \checkmark    &               &               &               &               &               \\ \hline
\cite{9223754}                      & \checkmark    &               &               &               &               &               &               &               \\ \hline
\cite{FANG20221}                    &               &               &               &               &               &               &               & \checkmark    \\ \hline
\cite{9399813}                      & \checkmark    &               & \checkmark    &               &               & \checkmark    &               &               \\ \hline
\cite{8832210}                      &               &               &               &               &               &               & \checkmark    &               \\ \hline
\cite{8994206}                      &               &               &               &               &               & \checkmark    &               &               \\ \hline
\cite{8733825}                      & \checkmark    &               &               &               &               &               &               &               \\ \hline
\cite{9274451}                      &               & \checkmark    &               &               &               &               &               &               \\ \hline
\cite{9293091}                      &               &               &               &               &               &               & \checkmark    &               \\ \hline 
\cite{8843900}                      &               &               &               &               & \checkmark    &               &               &               \\ \hline
\cite{8998397}                      &               &               & \checkmark    &               &               &               &               &               \\ \hline
\cite{9311394}                      &               &               & \checkmark    &               &               &               &               &               \\ \hline
\cite{9170905}                      &               &               & \checkmark    &               &               &               &               &               \\ \hline
\cite{8945913}                      &               & \checkmark    &               &               &               &               &               &               \\ \hline
\cite{10.48550/arxiv.2007.03856}    & \checkmark    &               &               &               &               &               &               &               \\ \hline
\cite{10.48550/arxiv.1912.04859}    & \checkmark    &               & \checkmark    &               &               &               &               &               \\ \hline
\cite{9321132}                      &               &               &               &               &               &               & \checkmark    &               \\ \hline
\cite{9079513}                      & \checkmark    &               &               &               &               & \checkmark    &               &               \\ \hline
\cite{app8122663}                   &               &               &               &               &               &               & \checkmark    &               \\ \hline
\cite{9347812}                      &               &               &               & \checkmark    &               &               &               &               \\ \hline
\cite{9134967}                      & \checkmark    &               &               &               &               &               &               &               \\ \hline
\cite{baffle}                       &               & \checkmark    &               &               &               &               &               &               \\ \hline
\cite{9292450}                      &               &               &               & \checkmark    &               &               &               &               \\ \hline
\cite{9210531}                      &               &               &               &               &               &               & \checkmark    &               \\ \hline
\cite{8894364}                      &               &               &               &               &               &               & \checkmark    &               \\ \hline
\cite{10.48550/arxiv.2112.07938}    & \checkmark    &               &               &               &               &               &               &               \\ \hline
\cite{demo}                         &               & \checkmark    &               &               &               &               &               &               \\ \hline
\cite{9233457}                      & \checkmark    &               &               &               &               &               &               &               \\ \hline
\cite{9170559}                      &               &               & \checkmark    &               &               & \checkmark    &               &               \\ \hline
\cite{pirate}                       &               &               &               &               &               &               & \checkmark    &               \\ \hline
\end{tabular}
\end{table}


As can be seen on \autoref{tab:consensus_mechanisms}, there is a high variety of consensus algorithms being used among implementations of BFL systems. Below is a summary of each consensus algorithm, as well as some of their characteristics.

\begin{itemize}
    \item \textit{Proof of Work (PoW).}
    
    \item \textit{Proof of Stake (PoS).}
    
    \item \textit{Proof of Authority (PoA).}
    
    \item \textit{Proof of Federated Learning (PoFL).}

    \item \textit{Practical Byzantine Fault Tolerance (PBFT).}
    
    \item \textit{Proof of Quality (PoQ).}

    \item \textit{Committee Based Consensus} \todo{isn't thius kinda PoA too?}
\end{itemize}

\subsection{Conclusions}

% Most implementations that used an already existing Blockchain platform went for Ethereum. However, a large proportion also implemented their own blockchain platform for BFL.