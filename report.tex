\documentclass[a4paper]{article}
\usepackage[utf8]{inputenc}
\usepackage{listings}
\usepackage{enumitem}
\lstset
{ %Formatting for code in appendix
    basicstyle=\footnotesize,
    numbers=left,
    showstringspaces=false,
    breaklines=true,
}

%\title{Sorting of Binary Search Trees in Coq}

\title{%
  Binary Search Trees in Coq \\
  \large Proving with Computer Assistance Assignment}
    
\author{Henrique Dias (1531484), Venislav Varbanov (XXXXXXX)}
\date{March 2021}

\begin{document}

\maketitle

%Do not make it too long. 15 pages is the absolute maximum but normally it should be much shorter. Keep in mind that longer does not mean better!
%What you should write:
%   Names and student numbers.
%   Explanation of the problem and your approach to it.
%   Description of the main definitions and the line of your proofs (e.g. sublemmas you used). If you had some alternative ideas to solve those problems describe them and explain your choice for the solution to this problem.
%   Write about your experience with the prover. What did you like, what you did not like etc.
%   Possibly add the Coq code as an appendix. (But note that you should deliver the Coq .v files separately anyway.)
%What you should not write
%   Do not unnecessarily repeat the code. Refer to appendix and quote the code only to illustrate something.
%   Do not write obvious things! Description of the proofs of the shape: "the goal is as follows so we apply this tactic and that is what we get..." are useless.


\section{Introduction}

The goal of this assignment was to formally define trees and binary search trees (BSTs) of natural numbers, define some of its common operations, prove their correctness, and prove that the leftmost node of a BST is its minimal element.

The implemented operations are:

\begin{itemize}[noitemsep]
    \item Sort: transforms a regular tree in a BST.
    \item Occurs: checks if a certain number occurs in a tree.
    \item Tree minimum: retrieves the minimum value stored in a tree.
    \item Left most item: retrieves the left most node's value of a tree.
    \item Insert: inserts a value in a BST.
    \item Search: similar to occurs, but leverages the fact that the given tree is a BST.
\end{itemize}

\section{Sorting of BSTs}

\section{Minimum of BSTs}

\appendix

\section{Code}

\lstinputlisting{bst.v}

\end{document}
